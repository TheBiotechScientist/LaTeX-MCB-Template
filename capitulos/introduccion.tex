% !TEX root = ..\main.tex

\chapter{Introducción}
\lettrine[lines=3]{L}{orem ipsum} dolor sit amet, consectetur adipisicing elit, sed do eiusmod tempor incididunt ut labore et dolore magna aliqua. Ut enim ad minim veniam, quis nostrud exercitation ullamco laboris nisi ut aliquip ex ea commodo consequat. Duis aute irure dolor in reprehenderit in voluptate velit esse cillum dolore eu fugiat nulla pariatur. Excepteur sint occaecat cupidatat non proident, sunt in culpa qui officia deserunt mollit anim id est laborum\citep{ejemplo01}.


\begin{equation*}
  \si{km\per h = m\per s}
\end{equation*}

\lipsum[2]\cite{ejemplo02}.

\section{La ecuación \texorpdfstring{$E=mc$}{E=mc}}
\lipsum[3]

\lipsum[4] Revisar la tabla \ref{tab:apnx:uno:sec1} del Apéndice \ref{apnx:uno}.

\lipsum[5-6]\citep{Dan,Baz}.

\section{Conversión de unidades}% \texorpdfstring{$\si{km\per h = \si{m\per s}}$}{kmh-ms}}
\lipsum[7]

\begin{figure}[!h]
  \center
  \includegraphics[width=0.5\textwidth]{workinprogress}
  \caption{Figura de ejemplo.}
  \label{fig:ejemplo1}
\end{figure}

\lipsum[8-9]

\begin{table}[!ht]
  \caption{Tabla de ejemplo.}
  \centering
  \begin{tabular}{ccc}

    \toprule
    \textbf{Columna 1} & \textbf{Columna 2} & \textbf{Columna 3}\\
    \midrule
    Dato 1             & Dato 2             & Dato 3\\
    Dato 1             & Dato 2             & Dato 3\\
    Dato 1             & Dato 2             & Dato 3\\
    Dato 1             & Dato 2             & Dato 3\\
    Dato 1             & Dato 2             & Dato 3\\
    Dato 1             & Dato 2             & Dato 3\\
    Dato 1             & Dato 2             & Dato 3\\
    Dato 1             & Dato 2             & Dato 3\\
    \bottomrule

  \end{tabular}
  \label{tab:ejemplo1}
\end{table}

\lipsum[10]

\section[Tablas con notas al pie]{Tablas con notas al pie usando el paquete\\ {\normalfont\large\texttt{threeparttable}}}
\lipsum[11]


\begin{table}[!ht]
  \caption{Tabla de ejemplo.}
  \centering
  \begin{threeparttable}
  \begin{tabular}{ccc}

    \toprule
    \textbf{Columna 1} & \textbf{Columna 2}\tnote{a} & \textbf{Columna 3}\\
    \midrule
    Dato 1             & Dato 2             & Dato 3\\
    Dato 1             & Dato 2             & Dato 3\\
    Dato 1             & Dato 2             & Dato 3\\
    Dato 1             & Dato 2             & Dato 3\\
    Dato 1             & Dato 2             & Dato 3\\
    Dato 1             & Dato 2             & Dato 3\\
    Dato 1             & Dato 2             & Dato 3\\
    Dato 1             & Dato 2             & Dato 3\tnote{b}\\
    \bottomrule

  \end{tabular}
  \begin{tablenotes}
    \item \emph{Nota:} Esta es una nota al final de la tabla usando el paquete {\small\texttt{threeparttable}} y utilizando símbolos con el paquete {\small\verb|siunitx|} como \si{\degree}$C$.
    \item [a] Esta referencia es de la columna 2.
    \item [b] Este dato es 3.
  \end{tablenotes}
  \end{threeparttable}
  \label{tab:ejemplo4}
  \end{table}


\begin{table}[!ht]
  \caption{En un lugar de La Mancha.}
  \centering
  \begin{tabular}{ccc}

  \toprule
  \textbf{Don Quijote} & \textbf{Columna 2} & \textbf{Columna 3}\\
  \midrule
  Dato 1             & Dato 2             & Dato 3\\
  Dato 1             & Dato 2             & Dato 3\\
  Dato 1             & Dato 2             & Dato 3\\
  Dato 1             & Dato 2             & Dato 3\\
  Dato 1             & Dato 2             & Dato 3\\
  Dato 1             & Dato 2             & Dato 3\\
  Dato 1             & Dato 2             & Dato 3\\
  \bottomrule

  \end{tabular}
  \label{tab:quijote}
\end{table}


\begin{table}[!ht]
  \caption{Don Quijote de La Mancha.}
  \centering
  \begin{threeparttable}
  \begin{tabular}{ccc}

  \toprule
  \textbf{La Mancha}\tnote{a} & \textbf{Columna 2} & \textbf{Columna 3}\\
  \midrule
  Dato 1             & Dato 2             & Dato 3\\
  Dato 1             & Dato 2             & Dato 3\\
  Dato 1             & Dato 2             & Dato 3\\
  Dato 1             & Dato 2             & Dato 3\\
  Dato 1             & Dato 2             & Dato 3\\
  Dato 1             & Dato 2             & Dato 3\\
  \bottomrule

  \end{tabular}
  \begin{tablenotes}
    \item[a] En un lugar de La Mancha.
  \end{tablenotes}
  \end{threeparttable}
  \label{tab:donquijote}
\end{table}


\section{Icluyendo código fuente}
Este es un ejemplo de código python en línea con el texto texto texto y más texto:

{\small\mint{python}{import numpy as np}}

Y aquí un ejemplo de código latex en línea con el texto:

{\small\mint{tex}{\documentclass[12pt,towside]{book}}}
{\small\mint{tex}{\usepackage[utf8]{inputenc}}}
